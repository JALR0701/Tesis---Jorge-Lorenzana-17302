\chapter{Modelo 3D}
Lo primero en realizar serán las modificaciones y arreglos al modelo mecánico del rostro animatrónico. Estas modificaciones incluyen la base del rostro animatrónico, por una que sea más resistente y más eficiente que la que se tiene actualmente. De esta manera se podrá probar el funcionamiento del mismo para saber que movimientos son posibles de realizar y así tener una idea de como tendrá que comportarse. Con las piezas diseñadas e impresas se procederá a desarmar el modelo y reemplazar todas las piezas defectuosas, esto incluye modelos 3D, tornillos, tuercas y todas las piezas mecánicas necesarias para la reconstrucción del rostro. Con el rostro reconstruido se pondrá a funcionar los servomotores para verificar que todo esté funcionando correctamente, que movimientos son capaces de realizar y si todos los componentes electrónicos están funcionando o si hay alguno que deba cambiarse.

\chapter{Modelo de reconocimiento}


\chapter{Detector de emociones}

\chapter{Respuesta a las emociones}

\chapter{GUI}